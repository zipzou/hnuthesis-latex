\documentclass[a4paper]{hnuthesis}
% 表格
\usepackage{longtable, multirow}
% 英文使用 Times 字体
\usepackage{times}
% 源代码
\usepackage{fancyvrb}
% 自定义列表样式
\usepackage{enumitem}
\usepackage{url}
\usepackage{amsmath}
\usepackage{amssymb}
\usepackage{moreverb}
\usepackage{txfonts}
\usepackage{mathcomp}
\usepackage{graphicx}
\usepackage{subfigure}
\usepackage[linesnumbered,boxed,ruled,vlined]{algorithm2e}
\usepackage{array}
\usepackage{multirow}

%%	added by Jiang
\usepackage{extarrows}	%使用长箭头
\usepackage{nomencl}	%与术语表有关的包
\usepackage{booktabs}
% \usepackage{ccmap}
\usepackage[numbers]{natbib}
\usepackage{color}
\usepackage{caption}
\usepackage{algorithmicx}
% \usepackage[chapter]{algorithm}
\usepackage[linesnumbered]{algorithm2e}
\usepackage{algpseudocode}
\usepackage{setspace}
\usepackage{pifont}
\usepackage{listings}
\defaultfontfeatures{Mapping=tex-text}
\setmainfont{Times New Roman}
\newfontfamily\fira{Fira Code}
\lstset{
    basicstyle          =   \footnotesize\fira,          % 基本代码风格
    keywordstyle        =   \bfseries\itshape,          % 关键字风格
    commentstyle        =   \color{commentgreen}\fira\itshape,  % 注释的风格,斜体
    stringstyle         =   \fira,  % 字符串风格
    flexiblecolumns,                % 别问为什么,加上这个
    numbers             =   left,   % 行号的位置在左边
    showspaces          =   false,  % 是否显示空格,显示了有点乱,所以不现实了
    numberstyle         =   \footnotesize\fira,    % 行号的样式,小五号,tt等宽字体
    showstringspaces    =   false,
    captionpos          =   t,      % 这段代码的名字所呈现的位置,t指的是top上面
    frame               =   b,   % 显示边框
    framerule       =   1pt
}

\lstdefinestyle{Python}{
    language        =   Python, % 语言选Python
    basicstyle      =   \footnotesize\fira,
    numberstyle     =   \footnotesize\fira,
    keywordstyle    =   \bfseries\color{blue},
    keywordstyle    =   [2] \bfseries\color{teal},
    stringstyle     =   \color{magenta},
    commentstyle    =   \color{commentgreen}\fira,
    breaklines      =   true,   % 自动换行,建议不要写太长的行
    columns         =   fixed,  % 如果不加这一句,字间距就不固定,很丑,必须加
    basewidth       =   0.5em,
    framerule       =   1pt
}

\title{论文标题}
\titleen{Paper Title with English}
\author{作者姓名}
\studentnum{作者学号}
\grade{班级}
\advisor{\kaishu XXX~~教授}

\major{\kaishu }
\researchfield{\kaishu }
\footdate{\kaishu 2021~年~5~月}
\submitdate{\kaishu 2021~年~5~月~23~日}
\defenddate{\kaishu 2021~年~5~月~23~日}
\institute{学院名称}
\instituteleader{XXX~~教授}

\begin{document}

\maketitle

\frontmatter

\pagestyle{plain}

\makepaperdeclare

\begin{abstract}
  这是中文摘要

  \keywords{中文、关键词1}
\end{abstract}

\begin{engabstract}

  This is Abstract with English
  
  \enkeywords{Keyword1, Keyword2}
\end{engabstract}

\addtocontents{toc}{\protect\setcounter{tocdepth}{-1}}
\tableofcontents
\addtocontents{toc}{\protect\setcounter{tocdepth}{3}}

\mainmatter

\chapter{如何使用}

本模板仅支持使用xelatex编译,用于适配Linux、macOS、Windows等多平台,为添加中文支持,本模板使用\lstinline{xeCJK}包构建中文。使用时,可根据示例文件\lstinline{hnu-paper.tex}修改内容即可。

本模板支持配置盲审模式,若需产生盲审送审版,可在文件\lstinline{hnu-paper.tex}头部\lstinline{\documentclass[review]}中添加选项为\lstinline{review},即可产生盲审版,去除所有作者、导师、学院等信息。正常模式下,去除参数\lstinline{review}即可完整显示。

\section{论文元信息设定}

论文元信息主要包括:\begin{enumerate}[itemsep=0pt, itemindent=12pt]
  \item 论文标题,指令:\lstinline{\title}
  \item 作者姓名,指令:\lstinline{\author}
  \item 作者学号,指令:\lstinline{\studentnum}
  \item 作者班级,指令:\lstinline{\grade}
  \item 学院名称,指令:\lstinline{\institute}
  \item 指导老师,指令:\lstinline{\advisor}
  \item 学院院长,指令:\lstinline{\instituteleader}
  \item 论文提交日期,指令:\lstinline{\submitdate}
\end{enumerate}
这些元信息,可通过在论文正文开始前逐个配置,并在\lstinline{\maketitle}后自动配置到论文封面。
\section{图片使用}
\figref{fig:logo}展示了图片引用方式,在引用图片时,可以采用命令\lstinline{\figref},并提供label作为参数,自动生成需要引用的图片编号。同理,\figref{fig:multi}为多张图片并列时的效果,并且子图将使用英文字母进行编号,并且通过子图的label标签,可以引用到子图,如\figref{fig:1}所示的内容。
\begin{figure}
  \centering
  \includegraphics[width=0.5\textwidth]{hnu-logo}
  \caption{大学logo}
  \label{fig:logo}
\end{figure}

\begin{figure}
  \centering
  \begin{minipage}[b]{0.48\textwidth}
    \centering
    \subfigure[并列图片1]{
      \includegraphics[width=0.5\columnwidth]{figs/hnu-logo.pdf}
      \label{fig:1}
    }
  \end{minipage}
  \begin{minipage}[b]{0.48\textwidth}
    \centering
    \subfigure[并列图片1]{
      \includegraphics[width=0.5\columnwidth]{hnu-title}
      \label{fig:2}
    }
  \end{minipage}
  \caption{并列图片示例}
  \label{fig:multi}
\end{figure}

\section{表格使用}

\begin{table}[htp]
  \centering
  \caption{表示例}
  \label{tab:table1}
  \begin{tabular}{ccc}
    \toprule
    \textbf{列1} & \textbf{列2} & \textbf{列3} \\
    \midrule
    1 & 2 & 3 \\
    4 & 5 & 6 \\
    9 & 8 & 7 \\
    \bottomrule
  \end{tabular}
\end{table}

\section{使用列表}

列表通常包含两种形式,有序列表和无序列表
\begin{enumerate}[itemindent=42pt, itemsep=0pt, label=(\arabic{*}), labelsep=30pt]
  \item 这是有序列表的形式
  \item 使用参数\lstinline{itemindent}可以更改缩进
  \item 使用参数\lstinline{iemsep}可以更改列表项之间的垂直间距
  \item 使用参数\lstinline{label}可以更改列表的编号风格
  \item 使用参数\lstinline{labelsep}可以更改编号标签与文本的间距
  \item\label{item:enum} 列表项同样可以使用\lstinline{\label}加标签引用,如\ref{item:enum}
\end{enumerate}


\section{公式使用}

  \subsection{行内公式}

  行内公式包裹在\lstinline{$$}内即可,如$c=a+b$。行内公式与文本同行,无法进行编号,并且当公式长度过长,可能出现无法正确换行的情况。当公式过长时应尽量使用块公式。
  同时,行内公式在使用求和上下标时,将与块公式不同:$\sum_{i=1}^n i$,当使用\lstinline{\limits}则可产生块公式相同效果:$\sum\limits_{i=1}^n i$。

  \subsection{块公式}

  求和公式:
  $$
  sum = \sum_{i=1}^n i
  $$
  该公式无法编号,也无法引用,当需要引用编号时,应该用环境形式编写,如公式~\eqref{eq:p}:
  \begin{equation}
    p = \left\{
      \begin{array}{cl} % c表示该列居中对齐,l表示左对齐
        50\%, & f < \bar{t}\\
        \frac{f-\bar{t}}{\bar{\tau} - \bar{t}} \times 50\% + 50\%, & \bar{t} \leq f < \bar{\tau} \\
        100\%, & \text{otherwise}
      \end{array}
    \right.
    \label{eq:p}
  \end{equation}


\chapter{如何管理文献}

在latex中,使用bibtex自动管理文献,而无需手动管理,如需要引用文献时,则只需要\cite{kopka1995guide},若需要引用作者,则只需要\citeauthor{DBLP:books/sp/Gliwa21},若需要按年份引用,则只需要\citeyear{DBLP:books/sp/Gliwa21}。
对于计算机专业,推荐使用dblp\footnote{\href{https://dblp.uni-trier.de/}{\url{https://dblp.uni-trier.de/}}}网站搜索并获得文献bibtex,中文文献可访问知网、万方、百度学术等数据库,获取文献信息或手动编写bibtex项,或通过Goole Scholar快速获取bibtex。
\chapter{绪论}
\section{研究背景}

这是二级标题,用于表示小节

\section{国内外研究现状}

这是三级标题,用于

\section{相关技术}

\section{论文组织结构}

\section{本章小节}
\chapter{第二章节}


\section{本章小结} 
\chapter{概要设计}

概要设计主要从宏观上,对系统的内容进行简要说明,通过UML图的形式,来表现系统的基本框架。

\section{本章小结}
\chapter{详细设计}

详细设计主要针对项目中的细节,以及概要设计的内容,详细地阐述系统的实现和设计过程
\input{sections/chapter5.tex}


\begin{conclusion}
  这是结论,主要描写本文取得的成果
\end{conclusion}

\bibliography{ref}
\addcontentsline{toc}{chapter}{参考文献}
\bibliographystyle{NJUthesis}

\begin{acknowledge}
  这里是致谢,一般是学校、导师、家人、同学
\end{acknowledge}

\end{document}