\chapter{如何使用}

本模板仅支持使用xelatex编译,用于适配Linux、macOS、Windows等多平台,为添加中文支持,本模板使用\lstinline{xeCJK}包构建中文。使用时,可根据示例文件\lstinline{hnu-paper.tex}修改内容即可。

本模板支持配置盲审模式,若需产生盲审送审版,可在文件\lstinline{hnu-paper.tex}头部\lstinline{\documentclass[review]}中添加选项为\lstinline{review},即可产生盲审版,去除所有作者、导师、学院等信息。正常模式下,去除参数\lstinline{review}即可完整显示。

\section{论文元信息设定}

论文元信息主要包括:\begin{enumerate}[itemsep=0pt, itemindent=12pt]
  \item 论文标题,指令:\lstinline{\title}
  \item 作者姓名,指令:\lstinline{\author}
  \item 作者学号,指令:\lstinline{\studentnum}
  \item 作者班级,指令:\lstinline{\grade}
  \item 学院名称,指令:\lstinline{\institute}
  \item 指导老师,指令:\lstinline{\advisor}
  \item 学院院长,指令:\lstinline{\instituteleader}
  \item 论文提交日期,指令:\lstinline{\submitdate}
\end{enumerate}
这些元信息,可通过在论文正文开始前逐个配置,并在\lstinline{\maketitle}后自动配置到论文封面。
\section{图片使用}
\figref{fig:logo}展示了图片引用方式,在引用图片时,可以采用命令\lstinline{\figref},并提供label作为参数,自动生成需要引用的图片编号。同理,\figref{fig:multi}为多张图片并列时的效果,并且子图将使用英文字母进行编号,并且通过子图的label标签,可以引用到子图,如\figref{fig:1}所示的内容。
\begin{figure}
  \centering
  \includegraphics[width=0.5\textwidth]{hnu-logo}
  \caption{大学logo}
  \label{fig:logo}
\end{figure}

\begin{figure}
  \centering
  \begin{minipage}[b]{0.48\textwidth}
    \centering
    \subfigure[并列图片1]{
      \includegraphics[width=0.5\columnwidth]{figs/hnu-logo.pdf}
      \label{fig:1}
    }
  \end{minipage}
  \begin{minipage}[b]{0.48\textwidth}
    \centering
    \subfigure[并列图片1]{
      \includegraphics[width=0.5\columnwidth]{hnu-title}
      \label{fig:2}
    }
  \end{minipage}
  \caption{并列图片示例}
  \label{fig:multi}
\end{figure}

\section{表格使用}

\begin{table}[htp]
  \centering
  \caption{表示例}
  \label{tab:table1}
  \begin{tabular}{ccc}
    \toprule
    \textbf{列1} & \textbf{列2} & \textbf{列3} \\
    \midrule
    1 & 2 & 3 \\
    4 & 5 & 6 \\
    9 & 8 & 7 \\
    \bottomrule
  \end{tabular}
\end{table}

\section{使用列表}

列表通常包含两种形式,有序列表和无序列表
\begin{enumerate}[itemindent=42pt, itemsep=0pt, label=(\arabic{*}), labelsep=30pt]
  \item 这是有序列表的形式
  \item 使用参数\lstinline{itemindent}可以更改缩进
  \item 使用参数\lstinline{iemsep}可以更改列表项之间的垂直间距
  \item 使用参数\lstinline{label}可以更改列表的编号风格
  \item 使用参数\lstinline{labelsep}可以更改编号标签与文本的间距
  \item\label{item:enum} 列表项同样可以使用\lstinline{\label}加标签引用,如\ref{item:enum}
\end{enumerate}


\section{公式使用}

  \subsection{行内公式}

  行内公式包裹在\lstinline{$$}内即可,如$c=a+b$。行内公式与文本同行,无法进行编号,并且当公式长度过长,可能出现无法正确换行的情况。当公式过长时应尽量使用块公式。
  同时,行内公式在使用求和上下标时,将与块公式不同:$\sum_{i=1}^n i$,当使用\lstinline{\limits}则可产生块公式相同效果:$\sum\limits_{i=1}^n i$。

  \subsection{块公式}

  求和公式:
  $$
  sum = \sum_{i=1}^n i
  $$
  该公式无法编号,也无法引用,当需要引用编号时,应该用环境形式编写,如公式~\eqref{eq:p}:
  \begin{equation}
    p = \left\{
      \begin{array}{cl} % c表示该列居中对齐,l表示左对齐
        50\%, & f < \bar{t}\\
        \frac{f-\bar{t}}{\bar{\tau} - \bar{t}} \times 50\% + 50\%, & \bar{t} \leq f < \bar{\tau} \\
        100\%, & \text{otherwise}
      \end{array}
    \right.
    \label{eq:p}
  \end{equation}

